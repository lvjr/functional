% -*- coding: utf-8 -*-
% !TEX program = lualatex
\documentclass[oneside]{book}

% -*- coding: utf-8 -*-
% !TEX program = lualatex

\usepackage[a4paper,margin=2.5cm]{geometry}

\newcommand*{\myversion}{2022E}
\newcommand*{\mydate}{Version \myversion\ (\the\year-\mylpad\month-\mylpad\day)}
\newcommand*{\mylpad}[1]{\ifnum#1<10 0\the#1\else\the#1\fi}

\setlength{\parindent}{0pt}
\setlength{\parskip}{4pt plus 1pt minus 1pt}

\usepackage{codehigh}

\colorlet{highback}{blue9}
%\CodeHigh{lite}
\CodeHigh{language=latex/latex2,style/main=highback,style/code=highback}
\NewCodeHighEnv{code}{style/main=gray9,style/code=gray9}
\NewCodeHighEnv{demo}{style/main=gray9,style/code=gray9,demo}

\usepackage{enumitem}

\NewDocumentCommand\MySubScript{m}{$_{#1}$}

\ExplSyntaxOn
\NewDocumentCommand\PrintVarList{m}{
  \clist_set:Nn \l_tmpa_clist {#1}
  \clist_map_inline:Nn \l_tmpa_clist
    {
      \token_to_str:N ##1 ~
    }
}
\NewDocumentCommand\RelaceChacters{m}{
  \tl_set:Nn \lTmpaTl {#1}
  \regex_replace_once:nnN { \_ } { \c{MySubScript} } \lTmpaTl
}
\NewDocumentCommand\RelaceUnderScoreAll{m}{
  \tl_set:Nn \lTmpaTl {#1}
  \regex_replace_all:nnN { \_ } { \c{_} } \lTmpaTl
  \lTmpaTl
}
\ExplSyntaxOff

\NewDocumentEnvironment{variable}{om}{
  \vspace{5pt}
  \begin{minipage}{\linewidth}
  \hrule\vspace{4pt}\obeylines%
  \begingroup
  \ttfamily\bfseries\color{azure3}
  \PrintVarList{#2}
  \endgroup
  \par\vspace{4pt}\hrule
  \end{minipage}\par\nopagebreak\vspace{4pt}
}{%
  \vspace{5pt}%
}

\NewDocumentEnvironment{function}{om}{
  \vspace{5pt}%
}{\vspace{5pt}}

\NewDocumentEnvironment{syntax}{}{%
  \begin{minipage}{\linewidth}
  \hrule\vspace{4pt}\obeylines%
}{%
  \par\vspace{4pt}\hrule
  \end{minipage}\par\nopagebreak\vspace{4pt}
}

\NewDocumentEnvironment{texnote}{}{}{}

\NewDocumentCommand\cs{O{}m}{%
  \texttt{\bfseries\color{purple3}\cBackslashStr\RelaceUnderScoreAll{#2}}%
}
\NewDocumentCommand\meta{m}{%
  \RelaceChacters{#1}%
  \textsl{$\langle$\ignorespaces\lTmpaTl\unskip$\rangle$}%
}
\NewDocumentCommand\Arg{m}{%
  \RelaceChacters{#1}%
  \texttt{\{}\textsl{$\langle$\ignorespaces\lTmpaTl\unskip$\rangle$}\texttt{\}}%
}

\NewDocumentCommand\pkg{m}{\textsf{#1}}

\NewDocumentCommand\nan{}{\texttt{NaN}}
\NewDocumentCommand\enquote{m}{``#1''}

\let\tn=\cs

\RenewDocumentCommand\emph{m}{%
  \underline{\textsl{#1}}%
}

\newenvironment{l3regex-syntax}
  {\begin{itemize}\def\\{\char`\\}\def\makelabel##1{\hss\llap{\ttfamily##1}}}
  {\end{itemize}}

\usepackage{shortvrb}

\NewDocumentCommand \MyMakeShortVerb {} {%
  \MakeShortVerb \|%
  \MakeShortVerb \"%
}
\NewDocumentCommand \MyDeleteShortVerb {} {%
  \DeleteShortVerb \|%
  \DeleteShortVerb \"%
}

\AtBeginDocument{\MyMakeShortVerb}
\AtEndDocument{\MyDeleteShortVerb}
\AddToHook{env/demohigh/before}{\MyDeleteShortVerb}
\AddToHook{env/demohigh/after}{\MyMakeShortVerb}

\usepackage{hologo}
\providecommand*\LuaTeX{\hologo{LuaTeX}}
\providecommand*\pdfTeX{\hologo{pdfTeX}}
\providecommand*\XeTeX{\hologo{XeTeX}}

\usepackage{amsmath}

\usepackage{hyperref}
\hypersetup{
  colorlinks=true,
  urlcolor=blue3,
  linkcolor=blue3,
}

\usepackage{functional}
%\Functional{scoping=false,tracing=true}

\CodeHigh{lite}

\begin{document}

\chapter{Strings (\texttt{Str})}

\section{Scratch Variables of Strings}

\begin{variable}{\lTmpaStr,\lTmpbStr,\lTmpcStr,\lTmpiStr,\lTmpjStr,\lTmpkStr}
Scratch strings for local assignment. These are never used by
the \verb!functional! package, and so are safe for use with any
function. However, they may be overwritten by other
code and so should only be used for short-term storage.
\end{variable}

\begin{variable}{\gTmpaStr,\gTmpbStr,\gTmpcStr,\gTmpiStr,\gTmpjStr,\gTmpkStr}
Scratch strings for global assignment. These are never used by
the \verb!functional! package, and so are safe for use with any
function. However, they may be overwritten by other
code and so should only be used for short-term storage.
\end{variable}

\section{Public Functions for Strings}

\begin{function}{\StrNew}
\begin{syntax}
\cs{StrNew} \meta{str var}
\end{syntax}
Creates a new \meta{str var} or raises an error if the name is
already taken. The declaration is global. The \meta{str var} is
initially empty.
\end{function}

\begin{function}{\StrVarLog}
\begin{syntax}
\cs{StrVarLog} \meta{str var}
\end{syntax}
Writes the content of the \meta{str var} in the log file.
For example
\begin{codehigh}
\StrSet\lTmpiStr{1234\abcd5678}
\StrVarLog\lTmpiStr
\end{codehigh}
\end{function}

\begin{function}{\StrVarShow}
\begin{syntax}
\cs{StrVarShow} \meta{str var}
\end{syntax}
Displays the content of the \meta{str var} on the terminal.
\end{function}

\begin{function}{\StrUse}
\begin{syntax}
\cs{StrUse} \meta{str var}
\end{syntax}
Recovers the content of a \meta{str var} and returns the value.
An error is raised if the variable
does not exist or if it is invalid. Note that it is possible to use
a \meta{str} directly without an accessor function.
\end{function}

\begin{function}{\StrSet}
\begin{syntax}
\cs{StrSet} \meta{str var} \Arg{token list}
\end{syntax}
Converts the \meta{token list} to a \meta{string}, and stores the
result in \meta{str var}. For example
\begin{demohigh}
\StrSet\lTmpiStr{\IntMathMult{4}{5}}
\StrUse\lTmpiStr
\end{demohigh}
\end{function}

\begin{function}{\StrClear}
\begin{syntax}
\cs{StrClear} \meta{str var}
\end{syntax}
Clears the content of the \meta{str var}. For example
\begin{demohigh}
\StrSet\lTmpjStr{One}
\StrClear\lTmpjStr
\StrSet\lTmpjStr{Two}
\StrUse\lTmpjStr
\end{demohigh}
\end{function}

\begin{function}{\StrPutLeft}
\begin{syntax}
\cs{StrPutLeft} \meta{str var} \Arg{token list}
\end{syntax}
Converts the \meta{token list} to a \meta{string}, and prepends the
result to \meta{str var}.  The current contents of the \meta{str
var} are not automatically converted to a string. For example
\begin{demohigh}
\StrSet\lTmpkStr{Functional}
\StrPutLeft\lTmpkStr{Hello}
\StrUse\lTmpkStr
\end{demohigh}
\end{function}

\begin{function}{\StrPutRight}
\begin{syntax}
\cs{StrPutRight} \meta{str var} \Arg{token list}
\end{syntax}
Converts the \meta{token list} to a \meta{string}, and appends the
result to \meta{str var}.  The current contents of the \meta{str
var} are not automatically converted to a string. For example
\begin{demohigh}
\StrSet\lTmpkStr{Functional}
\StrPutRight\lTmpkStr{World}
\StrUse\lTmpkStr
\end{demohigh}
\end{function}

\begin{function}{\StrVarIfEmpty,\StrVarIfEmptyTF}
\begin{syntax}
\cs{StrVarIfEmpty} \meta{str var}
\cs{StrVarIfEmptyTF} \meta{str var} \Arg{true code} \Arg{false code}
\end{syntax}
Tests if the \meta{string variable} is entirely empty
(\emph{i.e.} contains no characters at all). For example
\begin{demohigh}
\StrSet\lTmpaStr{abc}
\StrVarIfEmptyTF\lTmpaStr{\Result{Empty}}{\Result{NonEmpty}}
\StrClear\lTmpaStr
\StrVarIfEmptyTF\lTmpaStr{\Result{Empty}}{\Result{NonEmpty}}
\end{demohigh}
\end{function}

\begin{function}{\StrIfEq,\StrIfEqTF}
\begin{syntax}
\cs{StrIfEq} \Arg{tl_1} \Arg{tl_2}
\cs{StrIfEqTF} \Arg{tl_1} \Arg{tl_2} \Arg{true code} \Arg{false code}
\end{syntax}
Compares the two \meta{token lists} on a character by character
basis (namely after converting them to strings),
and is \texttt{true} if the two \meta{strings} contain the same
characters in the same order.
%Thus for example
%\begin{codehigh}
%\StrIfEq{abc}{\TlToStr{abc}}
%\end{codehigh}
%is logically \texttt{true}.
See \cs{TlIfEq} to compare
tokens (including their category codes) rather than characters.
For example
\begin{demohigh}
\StrIfEqTF{abc}{abc}{\Result{Yes}}{\Result{No}}
\StrIfEqTF{abc}{xyz}{\Result{Yes}}{\Result{No}}
\end{demohigh}
%\begin{demohigh}
%\StrSet\lTmpaStr{abc}
%\StrSet\lTmpbStr{abc}
%\StrSet\lTmpcStr{xyz}
%\StrIfEqTF{\lTmpaStr}{\lTmpbStr}{\Result{Yes}}{\Result{No}}
%\StrIfEqTF{\lTmpaStr}{\lTmpcStr}{\Result{Yes}}{\Result{No}}
%\StrIfEqTF{\StrUse\lTmpaStr}{\StrUse\lTmpbStr}{\Result{Yes}}{\Result{No}}
%\StrIfEqTF{\StrUse\lTmpaStr}{\StrUse\lTmpcStr}{\Result{Yes}}{\Result{No}}
%\end{demohigh}
\end{function}

\begin{function}{\StrVarIfEq,\StrVarIfEqTF}
\begin{syntax}
\cs{StrVarIfEq} \meta{str var_1} \meta{str var_2}
\cs{StrVarIfEqTF} \meta{str var_1} \meta{str var_2} \Arg{true code} \Arg{false code}
\end{syntax}
Compares the content of two \meta{str variables} and
is logically \texttt{true} if the two contain the same characters
in the same order.  See \cs{TlVarIfEq} to compare tokens
(including their category codes) rather than characters.
\begin{demohigh}
\StrSet\lTmpaStr{abc}
\StrSet\lTmpbStr{abc}
\StrSet\lTmpcStr{xyz}
\StrVarIfEqTF\lTmpaStr\lTmpbStr{\Result{Yes}}{\Result{No}}
\StrVarIfEqTF\lTmpaStr\lTmpcStr{\Result{Yes}}{\Result{No}}
\end{demohigh}
\end{function}

\end{document}
