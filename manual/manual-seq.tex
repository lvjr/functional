% -*- coding: utf-8 -*-
% !TEX program = lualatex
\documentclass[oneside]{book}

% -*- coding: utf-8 -*-
% !TEX program = lualatex

\usepackage[a4paper,margin=2.5cm]{geometry}

\newcommand*{\myversion}{2022E}
\newcommand*{\mydate}{Version \myversion\ (\the\year-\mylpad\month-\mylpad\day)}
\newcommand*{\mylpad}[1]{\ifnum#1<10 0\the#1\else\the#1\fi}

\setlength{\parindent}{0pt}
\setlength{\parskip}{4pt plus 1pt minus 1pt}

\usepackage{codehigh}

\colorlet{highback}{blue9}
%\CodeHigh{lite}
\CodeHigh{language=latex/latex2,style/main=highback,style/code=highback}
\NewCodeHighEnv{code}{style/main=gray9,style/code=gray9}
\NewCodeHighEnv{demo}{style/main=gray9,style/code=gray9,demo}

\usepackage{enumitem}

\NewDocumentCommand\MySubScript{m}{$_{#1}$}

\ExplSyntaxOn
\NewDocumentCommand\PrintVarList{m}{
  \clist_set:Nn \l_tmpa_clist {#1}
  \clist_map_inline:Nn \l_tmpa_clist
    {
      \token_to_str:N ##1 ~
    }
}
\NewDocumentCommand\RelaceChacters{m}{
  \tl_set:Nn \lTmpaTl {#1}
  \regex_replace_once:nnN { \_ } { \c{MySubScript} } \lTmpaTl
}
\NewDocumentCommand\RelaceUnderScoreAll{m}{
  \tl_set:Nn \lTmpaTl {#1}
  \regex_replace_all:nnN { \_ } { \c{_} } \lTmpaTl
  \lTmpaTl
}
\ExplSyntaxOff

\NewDocumentEnvironment{variable}{om}{
  \vspace{5pt}
  \begin{minipage}{\linewidth}
  \hrule\vspace{4pt}\obeylines%
  \begingroup
  \ttfamily\bfseries\color{azure3}
  \PrintVarList{#2}
  \endgroup
  \par\vspace{4pt}\hrule
  \end{minipage}\par\nopagebreak\vspace{4pt}
}{%
  \vspace{5pt}%
}

\NewDocumentEnvironment{function}{om}{
  \vspace{5pt}%
}{\vspace{5pt}}

\NewDocumentEnvironment{syntax}{}{%
  \begin{minipage}{\linewidth}
  \hrule\vspace{4pt}\obeylines%
}{%
  \par\vspace{4pt}\hrule
  \end{minipage}\par\nopagebreak\vspace{4pt}
}

\NewDocumentEnvironment{texnote}{}{}{}

\NewDocumentCommand\cs{O{}m}{%
  \texttt{\bfseries\color{purple3}\cBackslashStr\RelaceUnderScoreAll{#2}}%
}
\NewDocumentCommand\meta{m}{%
  \RelaceChacters{#1}%
  \textsl{$\langle$\ignorespaces\lTmpaTl\unskip$\rangle$}%
}
\NewDocumentCommand\Arg{m}{%
  \RelaceChacters{#1}%
  \texttt{\{}\textsl{$\langle$\ignorespaces\lTmpaTl\unskip$\rangle$}\texttt{\}}%
}

\NewDocumentCommand\pkg{m}{\textsf{#1}}

\NewDocumentCommand\nan{}{\texttt{NaN}}
\NewDocumentCommand\enquote{m}{``#1''}

\let\tn=\cs

\RenewDocumentCommand\emph{m}{%
  \underline{\textsl{#1}}%
}

\newenvironment{l3regex-syntax}
  {\begin{itemize}\def\\{\char`\\}\def\makelabel##1{\hss\llap{\ttfamily##1}}}
  {\end{itemize}}

\usepackage{shortvrb}

\NewDocumentCommand \MyMakeShortVerb {} {%
  \MakeShortVerb \|%
  \MakeShortVerb \"%
}
\NewDocumentCommand \MyDeleteShortVerb {} {%
  \DeleteShortVerb \|%
  \DeleteShortVerb \"%
}

\AtBeginDocument{\MyMakeShortVerb}
\AtEndDocument{\MyDeleteShortVerb}
\AddToHook{env/demohigh/before}{\MyDeleteShortVerb}
\AddToHook{env/demohigh/after}{\MyMakeShortVerb}

\usepackage{hologo}
\providecommand*\LuaTeX{\hologo{LuaTeX}}
\providecommand*\pdfTeX{\hologo{pdfTeX}}
\providecommand*\XeTeX{\hologo{XeTeX}}

\usepackage{amsmath}

\usepackage{hyperref}
\hypersetup{
  colorlinks=true,
  urlcolor=blue3,
  linkcolor=blue3,
}

\usepackage{functional}
%\Functional{scoping=false,tracing=true}

\CodeHigh{lite}

\begin{document}

\chapter{Sequences and Stacks (\texttt{Seq})}

\section{Constant and Scratch Sequences}

\begin{variable}{\cEmptySeq}
Constant that is always empty.
\end{variable}

\begin{variable}{\lTmpaSeq,\lTmpbSeq,\lTmpcSeq,\lTmpiSeq,\lTmpjSeq,\lTmpkSeq}
Scratch sequences for local assignment. These are never used by
the \verb!functional! package, and so are safe for use with any
function. However, they may be overwritten by other
code and so should only be used for short-term storage.
\end{variable}

\begin{variable}{\gTmpaSeq,\gTmpbSeq,\gTmpcSeq,\gTmpiSeq,\gTmpjSeq,\gTmpkSeq}
Scratch sequences for global assignment. These are never used by
the \verb!functional! package, and so are safe for use with any
function. However, they may be overwritten by other
code and so should only be used for short-term storage.
\end{variable}

\section{Creating and Using Sequences}

\begin{function}{\seqNew}
\begin{syntax}
\cs{seqNew} \meta{sequence}
\end{syntax}
Creates a new \meta{sequence} or raises an error if the name is
already taken. The declaration is global. The \meta{sequence}
initially contains no items.
\begin{codehigh}
\seqNew \lFooSomeSeq
\end{codehigh}
\end{function}

\begin{function}{\seqConstFromClist}
\begin{syntax}
\cs{seqConstFromClist} \meta{seq var} \Arg{comma-list}
\end{syntax}
Creates a new constant \meta{seq var} or raises an error if the name
is already taken. The \meta{seq var} is set globally to contain the
items in the \meta{comma list}.
\begin{codehigh}
\seqConstFromClist \cFooSomeSeq {one,two,three}
\end{codehigh}
\end{function}

\begin{function}{\seqVarJoin}
\begin{syntax}
\cs{seqVarJoin} \meta{seq var} \Arg{separator}
\end{syntax}
Returns the contents of the \meta{seq var}, with
the \meta{separator} between the items.  If the sequence has
a single item, it is returned with no \meta{separator},
and an empty sequence returns nothing.  An error is raised if
the variable does not exist or if it is invalid.
\begin{demohigh}
\seqSetSplit \lTmpaSeq {|} {a|b|c|{de}|f}
\seqVarJoin \lTmpaSeq { and }
\end{demohigh}
%\begin{texnote}
%The result is returned within the \tn{unexpanded}
%primitive (\cs{exp_not:n}), which means that the \meta{items}
%do not expand further when appearing in an \texttt{x}-type
%or \texttt{e}-type argument expansion.
%\end{texnote}
\end{function}

\begin{function}{\seqVarJoinExtended}
\begin{syntax}
\cs{seqVarJoinExtended} \meta{seq var} \Arg{separator between two} \Arg{separator between more than two} \Arg{separator between final two}
\end{syntax}
Returns the contents of the \meta{seq var}, with
the appropriate \meta{separator} between the items.  Namely, if the
sequence has more than two items, the \meta{separator between more
than two} is placed between each pair of items except the last,
for which the \meta{separator between final two} is used.  If the
sequence has exactly two items, then they are joined
with the \meta{separator between two} and returned. If the sequence has
a single item, it is returned, and an empty sequence
returns nothing.  An error is raised if the variable does
not exist or if it is invalid.
\begin{demohigh}
\seqSetSplit \lTmpaSeq {|} {a|b|c|{de}|f}
\seqVarJoinExtended \lTmpaSeq { and } {, } {, and }
\end{demohigh}
The first separator argument is not used in this case
because the sequence has more than $2$ items.
%\begin{texnote}
%The result is returned within the \tn{unexpanded}
%primitive (\cs{exp_not:n}), which means that the \meta{items}
%do not expand further when appearing in an \texttt{x}-type
%or \texttt{e}-type argument expansion.
%\end{texnote}
\end{function}

\section{Viewing Sequences}

\begin{function}{\seqVarLog}
\begin{syntax}
\cs{seqVarLog} \meta{sequence}
\end{syntax}
Writes the entries in the \meta{sequence} in the log file.
\begin{codehigh}
\seqVarLog \lFooSomeSeq
\end{codehigh}
\end{function}

\begin{function}{\seqVarShow}
\begin{syntax}
\cs{seqVarShow} \meta{sequence}
\end{syntax}
Displays the entries in the \meta{sequence} in the terminal.
\begin{codehigh}
\seqVarShow \lFooSomeSeq
\end{codehigh}
\end{function}

\section{Setting Sequences}

\begin{function}{\seqSetFromClist}
\begin{syntax}
\cs{seqSetFromClist} \meta{sequence} \meta{comma-list}
\end{syntax}
Converts the data in the \meta{comma list} into a \meta{sequence}:
the original \meta{comma list} is unchanged.
\begin{demohigh}
\seqSetFromClist \lTmpaSeq {one,two,three}
\seqVarJoin \lTmpaSeq { and }
\end{demohigh}
\end{function}

\begin{function}{\seqSetSplit}
\begin{syntax}
\cs{seqSetSplit} \meta{sequence} \Arg{delimiter} \Arg{token list}
\end{syntax}
Splits the \meta{token list} into \meta{items} separated
by \meta{delimiter}, and assigns the result to the \meta{sequence}.
Spaces on both sides of each \meta{item} are ignored,
then one set of outer braces is removed (if any);
this space trimming behaviour is identical to that of
\pkg{Clist} functions. Empty \meta{items} are preserved by
\cs{seqSetSplit}, and can be removed afterwards using
\cs{seqVarRemoveAll} \meta{sequence} \verb|{}|.
The \meta{delimiter} may not contain \verb|{|, \verb|}| or \verb|#|
(assuming \TeX{}'s normal category code r\'egime).
If the \meta{delimiter} is empty, the \meta{token list} is split
into \meta{items} as a \meta{token list}.
%See also \cs{seqSetSplitKeepSpaces}, which omits space stripping.
\begin{demohigh}
\seqSetSplit \lTmpaSeq {,} {1,2,3}
\seqVarJoin \lTmpaSeq { and }
\end{demohigh}
\end{function}

\begin{function}{\seqSetEq}
\begin{syntax}
\cs{seqSetEq} \meta{sequence_1} \meta{sequence_2}
\end{syntax}
Sets the content of \meta{sequence_1} equal to that of
\meta{sequence_2}.
\begin{demohigh}
\seqSetFromClist \lTmpaSeq {one,two,three,four}
\seqSetEq \lTmpbSeq \lTmpaSeq
\seqVarJoin \lTmpbSeq { and }
\end{demohigh}
\end{function}

\begin{function}{\seqClear}
\begin{syntax}
\cs{seqClear} \meta{sequence}
\end{syntax}
Clears all items from the \meta{sequence}.
\begin{codehigh}
\seqClear \lTmpaSeq
\end{codehigh}
\end{function}

\begin{function}{\seqClearNew}
\begin{syntax}
\cs{seqClearNew} \meta{sequence}
\end{syntax}
Ensures that the \meta{sequence} exists globally by applying
\cs{seqNew} if necessary, then applies \cs{seqClear} to leave
the \meta{sequence} empty.
\begin{demohigh}
\seqClearNew \lFooSomeSeq
\seqSetFromClist \lFooSomeSeq {one,two,three}
\seqVarJoin \lFooSomeSeq { and }
\end{demohigh}
\end{function}

\begin{function}{\seqConcat}
\begin{syntax}
\cs{seqConcat} \meta{sequence_1} \meta{sequence_2} \meta{sequence_3}
\end{syntax}
Concatenates the content of \meta{sequence_2} and \meta{sequence_3}
together and saves the result in \meta{sequence_1}. The items in
\meta{sequence_2} are placed at the left side of the new sequence.
\begin{demohigh}
\seqSetFromClist \lTmpbSeq {one,two}
\seqSetFromClist \lTmpcSeq {three,four}
\seqConcat \lTmpaSeq \lTmpbSeq \lTmpcSeq
\seqVarJoin \lTmpaSeq {, }
\end{demohigh}
\end{function}

%\begin{function}{\seqSetSplitKeepSpaces}
%\begin{syntax}
%\cs{seqSetSplitKeepSpaces} \meta{sequence} \Arg{delimiter} \Arg{token list}
%\end{syntax}
%Splits the \meta{token list} into \meta{items} separated
%by \meta{delimiter}, and assigns the result to the \meta{sequence}.
%One set of outer braces is removed (if any) but any surrounding spaces
%are retained: any braces \emph{inside} one or more spaces are
%therefore kept. Empty \meta{items} are preserved by
%\cs{seqSetSplitKeepSpaces}, and can be removed afterwards using
%\cs{seqVarRemoveAll} \meta{sequence} \verb|{}|.
%The \meta{delimiter} may not contain \verb|{|, \verb|}| or \verb|#|
%(assuming \TeX{}'s normal category code r\'egime).
%If the \meta{delimiter} is empty, the \meta{token list} is split
%into \meta{items} as a \meta{token list}.
%%See also \cs{seqSetSplit}, which removes spaces around the delimiters.
%\end{function}

\begin{function}{\seqPutLeft}
\begin{syntax}
\cs{seqPutLeft} \meta{sequence} \Arg{item}
\end{syntax}
Appends the \meta{item} to the left of the \meta{sequence}.
\begin{demohigh}
\seqSetFromClist \lTmpaSeq {one,two}
\seqPutLeft \lTmpaSeq {zero}
\seqVarJoin \lTmpaSeq { and }
\end{demohigh}
\end{function}

\begin{function}{\seqPutRight}
\begin{syntax}
\cs{seqPutRight} \meta{sequence} \Arg{item}
\end{syntax}
Appends the \meta{item} to the right of the \meta{sequence}.
\begin{demohigh}
\seqSetFromClist \lTmpaSeq {one,two}
\seqPutRight \lTmpaSeq {three}
\seqVarJoin \lTmpaSeq { and }
\end{demohigh}
\end{function}

\section{Modifying Sequences}

While sequences are normally used as ordered lists, it may be
necessary to modify the content. The functions here may be used
to update sequences, while retaining the order of the unaffected
entries.

\begin{function}{\seqVarRemoveDuplicates}
\begin{syntax}
\cs{seqVarRemoveDuplicates} \meta{sequence}
\end{syntax}
Removes duplicate items from the \meta{sequence}, leaving the
left most copy of each item in the \meta{sequence}.  The \meta{item}
comparison takes place on a token basis, as for \cs{tlIfEqTF}.
\begin{demohigh}
\seqSetFromClist \lTmpaSeq {one,two,one,two,three}
\seqVarRemoveDuplicates \lTmpaSeq
\seqVarJoin \lTmpaSeq {,}
\end{demohigh}
%\begin{texnote}
%This function iterates through every item in the \meta{sequence} and
%does a comparison with the \meta{items} already checked. It is therefore
%relatively slow with large sequences.
%\end{texnote}
\end{function}

\begin{function}{\seqVarRemoveAll}
\begin{syntax}
\cs{seqVarRemoveAll} \meta{sequence} \Arg{item}
\end{syntax}
Removes every occurrence of \meta{item} from the \meta{sequence}.
The \meta{item} comparison takes place on a token basis, as for \cs{tlIfEqTF}.
\begin{demohigh}
\seqSetFromClist \lTmpaSeq {one,two,one,two,three}
\seqVarRemoveAll \lTmpaSeq {two}
\seqVarJoin \lTmpaSeq {,}
\end{demohigh}
\end{function}

\begin{function}{\seqVarReverse}
\begin{syntax}
\cs{seqVarReverse} \meta{sequence}
\end{syntax}
Reverses the order of the items stored in the \meta{sequence}.
\begin{demohigh}
\seqSetFromClist \lTmpaSeq {one,two,one,two,three}
\seqVarReverse \lTmpaSeq
\seqVarJoin \lTmpaSeq {,}
\end{demohigh}
\end{function}

%\begin{function}{\seqVarSort}
%\begin{syntax}
%\cs{seqVarSort} \meta{sequence} \Arg{comparison code}
%\end{syntax}
%Sorts the items in the \meta{sequence} according to the
%\meta{comparison code}, and assigns the result to
%\meta{sequence}. The details of sorting comparison are
%described in Section \ref{sec:l3sort:mech}.
%\end{function}

%\begin{function}{\seqVarShuffle}
%\begin{syntax}
%\cs{seqVarShuffle} \meta{seq var}
%\end{syntax}
%Sets the \meta{seq var} to the result of placing the items of the
%\meta{seq var} in a random order.  Each item is (roughly) as likely
%to end up in any given position.
%\begin{texnote}
%For sequences with more than $13$ items or so, only a small
%proportion of all possible permutations can be reached, because
%the random seed \cs{sys_rand_seed:} only has $28$-bits.  The use
%of \tn{toks} internally means that sequences with more than
%$32767$ or $65535$ items (depending on the engine) cannot be
%shuffled.
%\end{texnote}
%\end{function}

\section{Working with the Contents of Sequences}

\begin{function}{\seqVarCount}
\begin{syntax}
\cs{seqVarCount} \meta{sequence}
\end{syntax}
Returns the number of items in the \meta{sequence}
as an \meta{integer denotation}. The total number of items
in a \meta{sequence} includes those which are empty and duplicates,
\emph{i.e.} every item in a \meta{sequence} is unique.
\begin{demohigh}
\seqSetFromClist \lTmpaSeq {one,two,three,four}
\seqVarCount \lTmpaSeq
\end{demohigh}
\end{function}

\begin{function}{\seqVarItem}
\begin{syntax}
\cs{seqVarItem} \meta{sequence} \Arg{integer expression}
\end{syntax}
Indexing items in the \meta{sequence} from $1$ at the top (left), this
function evaluates the \meta{integer expression} and returns the
appropriate item from the sequence. If the
\meta{integer expression} is negative, indexing occurs from the
bottom (right) of the sequence. If the \meta{integer expression}
is larger than the number of items in the \meta{sequence} (as
calculated by \cs{seqVarCount}) then the function returns nothing.
\begin{demohigh}
\seqSetFromClist \lTmpaSeq {one,two,three,four}
\tlSet \lTmpaTl {\seqVarItem \lTmpaSeq {3}}
\tlUse \lTmpaTl
\end{demohigh}
%\begin{texnote}
%The result is returned within the \tn{unexpanded}
%primitive (\cs{exp_not:n}), which means that the \meta{item}
%does not expand further when appearing in an \texttt{x}-type
%or \texttt{e}-type argument expansion.
%\end{texnote}
\end{function}

\begin{function}{\seqVarRandItem}
\begin{syntax}
\cs{seqVarRandItem} \meta{seq var}
\end{syntax}
Selects a pseudo-random item of the \meta{sequence}. If the
\meta{sequence} is empty the result is empty.
\begin{demohigh}
\seqSetFromClist \lTmpaSeq {one,two,three,four,five,six}
\tlSet \lTmpaTl {\seqVarRandItem \lTmpaSeq}
\tlUse \lTmpaTl
\tlSet \lTmpaTl {\seqVarRandItem \lTmpaSeq}
\tlUse \lTmpaTl
\end{demohigh}
%This is not available in older versions of \XeTeX{}.
%\begin{texnote}
%The result is returned within the \tn{unexpanded}
%primitive (\cs{exp_not:n}), which means that the \meta{item}
%does not expand further when appearing in an \texttt{x}-type
%or \texttt{e}-type argument expansion.
%\end{texnote}
\end{function}

\section{Sequences as Stacks}

Sequences can be used as stacks, where data is pushed to and popped
from the top of the sequence. (The left of a sequence is the top, for
performance reasons.) The stack functions for sequences are not
intended to be mixed with the general ordered data functions detailed
in the previous section: a sequence should either be used as an
ordered data type or as a stack, but not in both ways.

\begin{function}{\seqGet}
\begin{syntax}
\cs{seqGet} \meta{sequence} \meta{token list variable}
\end{syntax}
Reads the top item from a \meta{sequence} into the
\meta{token list variable} without removing it from the
\meta{sequence}. The \meta{token list variable} is assigned locally.
If \meta{sequence} is empty the \meta{token list variable} is set to
the special marker \cs{qNoValue}.
\begin{demohigh}
\seqSetFromClist \lTmpaSeq {two,three,four}
\seqGet \lTmpaSeq \lTmpaTl
\tlUse \lTmpaTl
\end{demohigh}
\end{function}

\begin{function}{\seqGetT,\seqGetF,\seqGetTF}
\begin{syntax}
\cs{seqGetT} \meta{sequence} \meta{token list variable} \Arg{true code}
\cs{seqGetF} \meta{sequence} \meta{token list variable} \Arg{false code}
\cs{seqGetTF} \meta{sequence} \meta{token list variable} \Arg{true code} \Arg{false code}
\end{syntax}
If the \meta{sequence} is empty, leaves the \meta{false code} in the
input stream.  The value of the \meta{token list variable} is
not defined in this case and should not be relied upon.  If the
\meta{sequence} is non-empty, stores the top item from a
\meta{sequence} in the \meta{token list variable} without removing it from
the \meta{sequence}. The \meta{token list variable} is assigned locally.
\begin{demohigh}
\seqSetFromClist \lTmpaSeq {two,three,four}
\seqGetTF \lTmpaSeq \lTmpaTl {\prgReturn{Yes}} {\prgReturn{No}}
\end{demohigh}
\end{function}

\begin{function}{\seqPop}
\begin{syntax}
\cs{seqPop} \meta{sequence} \meta{token list variable}
\end{syntax}
Pops the top item from a \meta{sequence} into the
\meta{token list variable}. the \meta{token list variable} is assigned locally.
If \meta{sequence} is empty the \meta{token list variable} is set to
the special marker \cs{qNoValue}.
\begin{demohigh}
\seqSetFromClist \lTmpaSeq {two,three,four}
\seqPop \lTmpaSeq \lTmpaTl
\seqVarJoin \lTmpaSeq {,}
\end{demohigh}
\end{function}

\begin{function}{\seqPopT,\seqPopF,\seqPopTF}
\begin{syntax}
\cs{seqPopT} \meta{sequence} \meta{token list variable} \Arg{true code}
\cs{seqPopF} \meta{sequence} \meta{token list variable} \Arg{false code}
\cs{seqPopTF} \meta{sequence} \meta{token list variable} \Arg{true code} \Arg{false code}
\end{syntax}
If the \meta{sequence} is empty, leaves the \meta{false code} in the
input stream.  The value of the \meta{token list variable} is
not defined in this case and should not be relied upon.  If the
\meta{sequence} is non-empty, pops the top item from the \meta{sequence}
in the \meta{token list variable}, \emph{i.e.} removes the item from the
\meta{sequence}. The \meta{token list variable} is assigned locally.
\begin{demohigh}
\seqPopTF \cEmptySeq \lTmpaTl {\prgReturn{Yes}} {\prgReturn{No}}
\end{demohigh}
\end{function}

\begin{function}{\seqPush}
\begin{syntax}
\cs{seqPush} \meta{sequence} \Arg{item}
\end{syntax}
Adds the \Arg{item} to the top of the \meta{sequence}.
\begin{demohigh}
\seqSetFromClist \lTmpaSeq {two,three,four}
\seqPush \lTmpaSeq {one}
\seqVarJoin \lTmpaSeq {|}
\end{demohigh}
You can only push one item to the \meta{sequence} with \cs{seqPush},
which is different from \cs{ClistPush}.
\end{function}

\section{Recovering Items from Sequences}

Items can be recovered from either the left or the right of sequences.
For implementation reasons, the actions at the left of the sequence are
faster than those acting on the right. These functions all assign the
recovered material locally.

\begin{function}{\seqGetLeft}
\begin{syntax}
\cs{seqGetLeft} \meta{sequence} \meta{token list variable}
\end{syntax}
Stores the left-most item from a \meta{sequence} in the
\meta{token list variable} without removing it from the
\meta{sequence}. The \meta{token list variable} is assigned locally.
If \meta{sequence} is empty the \meta{token list variable}
is set to the special marker \cs{qNoValue}.
\begin{demohigh}
\seqSetFromClist \lTmpaSeq {two,three,four}
\seqGetLeft \lTmpaSeq \lTmpaTl
\tlUse \lTmpaTl
\end{demohigh}
\end{function}

\begin{function}{\seqGetLeftT,\seqGetLeftF,\seqGetLeftTF}
\begin{syntax}
\cs{seqGetLeftT} \meta{sequence} \meta{token list variable} \Arg{true code}
\cs{seqGetLeftF} \meta{sequence} \meta{token list variable} \Arg{false code}
\cs{seqGetLeftTF} \meta{sequence} \meta{token list variable} \Arg{true code} \Arg{false code}
\end{syntax}
If the \meta{sequence} is empty, leaves the \meta{false code} in the
input stream.  The value of the \meta{token list variable} is
not defined in this case and should not be relied upon.  If the
\meta{sequence} is non-empty, stores the left-most item from the
\meta{sequence}
in the \meta{token list variable} without removing it from the
\meta{sequence}, then leaves the \meta{true code} in the input stream.
The \meta{token list variable} is assigned locally.
\begin{demohigh}
\seqSetFromClist \lTmpaSeq {two,three,four}
\seqGetLeftTF \lTmpaSeq \lTmpaTl {\prgReturn{Yes}} {\prgReturn{No}}
\end{demohigh}
\end{function}

\begin{function}{\seqGetRight}
\begin{syntax}
\cs{seqGetRight} \meta{sequence} \meta{token list variable}
\end{syntax}
Stores the right-most item from a \meta{sequence} in the
\meta{token list variable} without removing it from the
\meta{sequence}. The \meta{token list variable} is assigned locally.
If \meta{sequence} is empty the \meta{token list variable}
is set to the special marker \cs{qNoValue}.
\begin{demohigh}
\seqSetFromClist \lTmpaSeq {two,three,four}
\seqGetRight \lTmpaSeq \lTmpaTl
\tlUse \lTmpaTl
\end{demohigh}
\end{function}

\begin{function}{\seqGetRightT,\seqGetRightF,\seqGetRightTF}
\begin{syntax}
\cs{seqGetRightT} \meta{sequence} \meta{token list variable} \Arg{true code}
\cs{seqGetRightF} \meta{sequence} \meta{token list variable} \Arg{false code}
\cs{seqGetRightTF} \meta{sequence} \meta{token list variable} \Arg{true code} \Arg{false code}
\end{syntax}
If the \meta{sequence} is empty, leaves the \meta{false code} in the
input stream.  The value of the \meta{token list variable} is
not defined in this case and should not be relied upon.  If the
\meta{sequence} is non-empty, stores the right-most item from the
\meta{sequence}
in the \meta{token list variable} without removing it from the
\meta{sequence}, then leaves the \meta{true code} in the input stream.
The \meta{token list variable} is assigned locally.
\begin{demohigh}
\seqSetFromClist \lTmpaSeq {two,three,four}
\seqGetRightTF \lTmpaSeq \lTmpaTl {\prgReturn{Yes}} {\prgReturn{No}}
\end{demohigh}
\end{function}

\begin{function}{\seqPopLeft}
\begin{syntax}
\cs{seqPopLeft} \meta{sequence} \meta{token list variable}
\end{syntax}
Pops the left-most item from a \meta{sequence} into the
\meta{token list variable}, \emph{i.e.} removes the item from the
sequence and stores it in the \meta{token list variable}.
The assignment of the \meta{token list variable} is local.
If \meta{sequence} is empty the \meta{token list variable}
is set to the special marker \cs{qNoValue}.
\begin{demohigh}
\seqSetFromClist \lTmpaSeq {two,three,four}
\seqPopLeft \lTmpaSeq \lTmpaTl
\seqVarJoin \lTmpaSeq {,}
\end{demohigh}
\end{function}

\begin{function}{\seqPopLeftT,\seqPopLeftF,\seqPopLeftTF}
\begin{syntax}
\cs{seqPopLeftT} \meta{sequence} \meta{token list variable} \Arg{true code}
\cs{seqPopLeftF} \meta{sequence} \meta{token list variable} \Arg{false code}
\cs{seqPopLeftTF} \meta{sequence} \meta{token list variable} \Arg{true code} \Arg{false code}
\end{syntax}
If the \meta{sequence} is empty, leaves the \meta{false code} in the
input stream.  The value of the \meta{token list variable} is
not defined in this case and should not be relied upon.  If the
\meta{sequence} is non-empty, pops the left-most item from the \meta{sequence}
in the \meta{token list variable}, \emph{i.e.} removes the item from the
\meta{sequence}, then leaves the \meta{true code} in the input stream.
The \meta{token list variable} is assigned locally.
\begin{demohigh}
\seqPopLeftTF \cEmptySeq \lTmpaTl {\prgReturn{Yes}} {\prgReturn{No}}
\end{demohigh}
\end{function}

\begin{function}{\seqPopRight}
\begin{syntax}
\cs{seqPopRight} \meta{sequence} \meta{token list variable}
\end{syntax}
Pops the right-most item from a \meta{sequence} into the
\meta{token list variable}, \emph{i.e.} removes the item from the
sequence and stores it in the \meta{token list variable}.
The assignment of the \meta{token list variable} is local.
If \meta{sequence} is empty the \meta{token list variable}
is set to the special marker \cs{qNoValue}.
\begin{demohigh}
\seqSetFromClist \lTmpaSeq {two,three,four}
\seqPopRight \lTmpaSeq \lTmpaTl
\seqVarJoin \lTmpaSeq {,}
\end{demohigh}
\end{function}

\begin{function}{\seqPopRightT,\seqPopRightF,\seqPopRightTF}
\begin{syntax}
\cs{seqPopRightT} \meta{sequence} \meta{token list variable} \Arg{true code}
\cs{seqPopRightF} \meta{sequence} \meta{token list variable} \Arg{false code}
\cs{seqPopRightTF} \meta{sequence} \meta{token list variable} \Arg{true code} \Arg{false code}
\end{syntax}
If the \meta{sequence} is empty, leaves the \meta{false code} in the
input stream.  The value of the \meta{token list variable} is
not defined in this case and should not be relied upon.  If the
\meta{sequence} is non-empty, pops the right-most item from the \meta{sequence}
in the \meta{token list variable}, \emph{i.e.} removes the item from the
\meta{sequence}, then leaves the \meta{true code} in the input stream.
The \meta{token list variable} is assigned locally.
\begin{demohigh}
\seqPopRightTF \cEmptySeq \lTmpaTl {\prgReturn{Yes}} {\prgReturn{No}}
\end{demohigh}
\end{function}

\section{Mapping over Sequences}

%All mappings are done at the current group level, \emph{i.e.} any
%local assignments made by the \meta{function} or \meta{code} discussed
%below remain in effect after the loop.

%\begin{function}{\seqVarMapFunction}
%\begin{syntax}
%\cs{seqVarMapFunction} \meta{sequence} \meta{function}
%\end{syntax}
%Applies \meta{function} to every \meta{item} stored in the
%\meta{sequence}. The \meta{function} will receive one argument for
%each iteration. The \meta{items} are returned from left to right.
%To pass further arguments to the \meta{function}, see \cs{seqVarMapTokens}.
%The function \cs{seqVarMapInline} is faster than
%\cs{seqVarMapFunction} for sequences with more than about $10$ items.
%\end{function}

\begin{function}{\seqVarMapInline}
\begin{syntax}
\cs{seqVarMapInline} \meta{sequence} \Arg{inline function}
\end{syntax}
Applies \meta{inline function} to every \meta{item} stored
within the \meta{sequence}. The \meta{inline function} should
consist of code which will receive the \meta{item} as \verb|#1|.
The \meta{items} are returned from left to right.
\begin{demohigh}
\IgnoreSpacesOn
\seqSetFromClist \lTmpkSeq {one,two,three}
\tlClear \lTmpaTl
\seqVarMapInline \lTmpkSeq {
  \tlPutRight \lTmpaTl {(#1)}
}
\tlUse \lTmpaTl
\IgnoreSpacesOff
\end{demohigh}
\end{function}

%\begin{function}{\seqVarMapTokens}
%\begin{syntax}
%\cs{seqVarMapTokens} \meta{sequence} \Arg{code}
%\end{syntax}
%Analogue of \cs{seqVarMapFunction} which maps several tokens
%instead of a single function.  The \meta{code} receives each item in
%the \meta{sequence} as a trailing brace group. For instance,
%\begin{verbatim}
%\seqVarMapTokens \lMySeq { \prgReplicate { 2 } }
%\end{verbatim}
%expands to twice each item in the \meta{sequence}: for each item in
%\verb|\l_my_seq| the function \cs{PrgReplicate} receives \verb|2| and
%\meta{item} as its two arguments.  The function
%\cs{seqVarMapInline} is typically faster but it is not expandable.
%\end{function}

\begin{function}{\seqVarMapVariable}
\begin{syntax}
\cs{seqVarMapVariable} \meta{sequence} \meta{variable} \Arg{code}
\end{syntax}
Stores each \meta{item} of the \meta{sequence} in turn in the (token
list) \meta{variable} and applies the \meta{code}.  The \meta{code}
will usually make use of the \meta{variable}, but this is not
enforced.  The assignments to the \meta{variable} are local.  Its
value after the loop is the last \meta{item} in the \meta{sequence},
or its original value if the \meta{sequence} is empty.  The
\meta{items} are returned from left to right.
\begin{demohigh}
\IgnoreSpacesOn
\intZero \lTmpaInt
\seqSetFromClist \lTmpaSeq {1,3,7}
\seqVarMapVariable \lTmpaSeq \lTmpiTl {
  \intAdd \lTmpaInt {\lTmpiTl*\lTmpiTl}
}
\intUse \lTmpaInt
\IgnoreSpacesOff
\end{demohigh}
\end{function}

%\begin{function}{\seqVarMapIndexedFunction}
%\begin{syntax}
%\cs{seqVarMapIndexedFunction} \meta{seq var} \meta{function}
%\end{syntax}
%Applies \meta{function} to every entry in the \meta{sequence
%variable}.  The \meta{function} should have signature |:nn|.  It
%receives two arguments for each iteration: the \meta{index} (namely
%\verb|1| for the first entry, then \verb|2| and so on) and the \meta{item}.
%\end{function}

%\begin{function}{\seqVarMapIndexedInline}
%\begin{syntax}
%\cs{seqVarMapIndexedInline} \meta{seq var} \Arg{inline function}
%\end{syntax}
%Applies \meta{inline function} to every entry in the \meta{sequence
%variable}.  The \meta{inline function} should consist of code which
%receives the \meta{index} (namely \verb|1| for the first entry, then \verb|2|
%and so on) as \verb|#1| and the \meta{item} as \verb|#2|.
%\end{function}

%\begin{function}{\seqMapBreak}
%\begin{syntax}
%\cs{seqMapBreak}
%\end{syntax}
%Used to terminate a seq map function before all
%entries in the \meta{sequence} have been processed. This
%normally takes place within a conditional statement, for example
%\begin{verbatim}
%\seq_map_inline:Nn \l_my_seq
%{
%\str_if_eq:nnTF { #1 } { bingo }
%{ \seq_map_break: }
%{
%Do something useful
%}
%}
%\end{verbatim}
%Use outside of a seq map scenario leads to low
%level \TeX{} errors.
%\begin{texnote}
%When the mapping is broken, additional tokens may be inserted
%before further items are taken
%from the input stream. This depends on the design of the mapping
%function.
%\end{texnote}
%\end{function}

%\begin{function}{\seqMapBreakDo}
%\begin{syntax}
%\cs{seqMapBreakDo} \Arg{code}
%\end{syntax}
%Used to terminate a seq map function before all
%entries in the \meta{sequence} have been processed, inserting
%the \meta{code} after the mapping has ended. This
%normally takes place within a conditional statement, for example
%\begin{verbatim}
%\seq_map_inline:Nn \l_my_seq
%{
%\str_if_eq:nnTF { #1 } { bingo }
%{ \seq_map_break:n { <code> } }
%{
%Do something useful
%}
%}
%\end{verbatim}
%Use outside of a seq map scenario leads to low
%level \TeX{} errors.
%\begin{texnote}
%When the mapping is broken, additional tokens may be inserted
%before the \meta{code} is
%inserted into the input stream.
%This depends on the design of the mapping function.
%\end{texnote}
%\end{function}

%\begin{function}
%{\seq_set_map:NNn, \seq_gset_map:NNn}
%\begin{syntax}
%\cs{seq_set_map:NNn} \meta{sequence_1} \meta{sequence_2} \Arg{inline function}
%\end{syntax}
%Applies \meta{inline function} to every \meta{item} stored
%within the \meta{sequence_2}. The \meta{inline function} should
%consist of code which will receive the \meta{item} as \verb|#1|.
%The sequence resulting applying \meta{inline function} to each
%\meta{item} is assigned to \meta{sequence_1}.
%\begin{texnote}
%Contrarily to other mapping functions, \cs{seq_map_break:} cannot
%be used in this function, and would lead to low-level \TeX{} errors.
%\end{texnote}
%\end{function}

%\begin{function}[added = 2020-07-16]
%{\seq_set_map_x:NNn, \seq_gset_map_x:NNn}
%\begin{syntax}
%\cs{seq_set_map_x:NNn} \meta{sequence_1} \meta{sequence_2} \Arg{inline function}
%\end{syntax}
%Applies \meta{inline function} to every \meta{item} stored
%within the \meta{sequence_2}. The \meta{inline function} should
%consist of code which will receive the \meta{item} as \verb|#1|.
%The sequence resulting from \texttt{x}-expanding
%\meta{inline function} applied to each \meta{item}
%is assigned to \meta{sequence_1}. As such, the code
%in \meta{inline function} should be expandable.
%\begin{texnote}
%Contrarily to other mapping functions, \cs{seq_map_break:} cannot
%be used in this function, and would lead to low-level \TeX{} errors.
%\end{texnote}
%\end{function}

\section{Sequence Conditionals}

\begin{function}{\seqIfExist,\seqIfExistT,\seqIfExistF,\seqIfExistTF}
\begin{syntax}
\cs{seqIfExist} \meta{sequence}
\cs{seqIfExistT} \meta{sequence} \Arg{true code}
\cs{seqIfExistF} \meta{sequence} \Arg{false code}
\cs{seqIfExistTF} \meta{sequence} \Arg{true code} \Arg{false code}
\end{syntax}
Tests whether the \meta{sequence} is currently defined.  This does not
check that the \meta{sequence} really is a sequence variable.
\begin{demohigh}
\seqIfExistTF \lTmpaSeq {\prgReturn{Yes}} {\prgReturn{No}}
\seqIfExistTF \lFooUndefinedSeq {\prgReturn{Yes}} {\prgReturn{No}}
\end{demohigh}
\end{function}

\begin{function}{\seqVarIfEmpty,\seqVarIfEmptyT,\seqVarIfEmptyF,\seqVarIfEmptyTF}
\begin{syntax}
\cs{seqVarIfEmpty} \meta{sequence}
\cs{seqVarIfEmptyT} \meta{sequence} \Arg{true code}
\cs{seqVarIfEmptyF} \meta{sequence} \Arg{false code}
\cs{seqVarIfEmptyTF} \meta{sequence} \Arg{true code} \Arg{false code}
\end{syntax}
Tests if the \meta{sequence} is empty (containing no items).
\begin{demohigh}
\seqSetFromClist \lTmpaSeq {one,two}
\seqVarIfEmptyTF \lTmpaSeq {\prgReturn{Empty}} {\prgReturn{NonEmpty}}
\seqClear \lTmpaSeq
\seqVarIfEmptyTF \lTmpaSeq {\prgReturn{Empty}} {\prgReturn{NonEmpty}}
\end{demohigh}
\end{function}

\begin{function}{\seqVarIfIn,\seqVarIfInT,\seqVarIfInF,\seqVarIfInTF}
\begin{syntax}
\cs{seqVarIfIn} \meta{sequence} \Arg{item}
\cs{seqVarIfInT} \meta{sequence} \Arg{item} \Arg{true code}
\cs{seqVarIfInF} \meta{sequence} \Arg{item} \Arg{false code}
\cs{seqVarIfInTF} \meta{sequence} \Arg{item} \Arg{true code} \Arg{false code}
\end{syntax}
Tests if the \meta{item} is present in the \meta{sequence}.
\begin{demohigh}
\seqSetFromClist \lTmpaSeq {one,two}
\seqVarIfInTF \lTmpaSeq {one} {\prgReturn{Yes}} {\prgReturn{Not}}
\seqVarIfInTF \lTmpaSeq {three} {\prgReturn{Yes}} {\prgReturn{Not}}
\end{demohigh}
\end{function}

\end{document}
